\begin{tcolorbox}
	In diesem Glossar können Akronyme und abkürzende Schreibweisen aufgelistet werden. 
	Alle verwendeten Abkürzungen innerhalb des Projekts müssen hier erläutert werden.
\end{tcolorbox}

\begin{table}[h]
	\centering
	\begin{tabularx}{\textwidth}{X X}
		\rowcolor[HTML]{C0C0C0} 
		\textbf{Abkürzung} & \textbf{Beschreibung} \\
		Zähler & Bei dem Zähler handelt es sich um den Überbegriff für Gas-, Strom- und Wasserzähler, die den Verbrauch der jeweils namensgebenden Ressource messen. \\
		\rowcolor[HTML]{E7E7E7} 
		FAB (Floating Action Button) & Der FAB ist ein Knopf, der in der unteren rechten Ecke einer Android-App sitzt und Zugriff zu essentiellen Funktionen ermöglicht. \\
		Administrator (Admin) & Der Administrator vornehmlich ein Mitarbeiter von “adesso energy” hat übergeordnete Zugriffsrechte, die es ihm erlauben auf Kundendaten zuzugreifen, diese zu ändern und Kontaktformulare zu bearbeiten. Weiterhin kann er den Zeitraum anpassen innerhalb dem Nutzer keine weiteren Zählerstände hochladen können. Der Administrator interagiert dabei ausschließlich mit der Website. \\
		\rowcolor[HTML]{E7E7E7} 
		Benutzer (User) & Als Benutzer werden adesso-energy-Kunden bezeichnet. Sie haben die Möglichkeit über die App oder Website auf ihre Nutzerdaten zuzugreifen und aktuelle Zählerstände über die in der App zur Verfügung gestellte Funktion hochzuladen. Der Benutzer kann sowohl mit der App als auch mit der Website interagieren. \\
		Aktualisierungswartezeit & Die von dem Administrator eingestellte Zeit, die ein Nutzer warten muss bis er nach dem Hochladen eines Bildes erneut eins hochladen kann. 
	\end{tabularx}
	\caption{Glossar}
	\label{table:glossar}
\end{table}
