\begin{tcolorbox}
In diesem Kapitel werden die folgenden drei Punkte erläutert:
\begin{enumerate}
	\item \textit{Anwendungsgebiete:} Was ist der Zweck des Produkts?
	\item \textit{Zielgruppen:} Für welche Benutzer (oder auch Rollen) ist das Produkt bestimmt?
	Welche Qualifikationen brauchen die Personen?
	\item \textit{Betriebsbedingungen:} Automatische oder manuelle Datensicherung? 	Autonomer oder beobachtender Betrieb? 	
\end{enumerate}

\noindent Die einzelnen Teile des Produkteinsatzes werden üblicherweise als Fließtexte geschrieben.
\end{tcolorbox}
Dieses Kapitel beschreibt die Rahmenbedingungen des Produkteinsatzes. Also die Anwendungsgebiete, die die genauer bestimmen, was der Zweck des Produktes ist, die Zielgruppen für die das Produkt entwickelt wird
und die Betriebsbedingungen, also die Umstände unter denen eine Firma das Produkt benutzen und warten kann.

\section{Anwendungsgebiete}
	Um den aktuellen Zählerständ an den Energieanbieter zu schicken mussten bisher die Stände abgelesen und per Post verschickt 		  
	werden. \\
	Das Produkt vereinfacht diesen Ablauf, indem es Kund*innen die Möglichkeit bietet in derr App ein Foto ihres Zählers hochzuladen. Dieses wird dann 	
	ausgewertet und der Zählerstand  automatisch übermittelt.
	Wenn die App nicht benutzt wird können die Zählerstände auch manuell auf der Website eingegeben werden.\\\\
	Die Administriernden können die Daten der betreuten Kund*innen verwalten und die Zählerstände einsehen.
\section{Zielgruppen}
	Die App und Website sind für die Kund*innen des Energieanbieters konzipiert. Das Interface und die Verwendung ist einfach gehalten, sodass auch Nutzende mit wenig
	Kentnissen diese benutzen können.
	
	Auch Administrierende benötigen keine besonderen Vorkenntnisse, um die Webseitenansicht zur Verwaltung der Daten zu nutzen.
	Grundlegende Kenntnisse über die Benutzung von Computern genügen hierfür.
	
\section{Betriebsbedingungen}
	Die Persistierung der Datenbank erfolgt automatisch. Der Betrieb der Datenbanken sollte regelmäßig von einem Mitarbeitenden, der dafür eingearbeitet wurde,  überwacht werden.
	Die Anwendung erstellt keine Datensicherungen. Wenn diese gewünscht sind sollten sie also manuell erfolgen.
	
