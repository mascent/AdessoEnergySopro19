\section{Dokumentaufbau}\label{sec:dokumentaufbau}
\begin{tcolorbox}
	Inhalt und Struktur des vorliegenden Dokuments skizzieren (Fließtext).
\end{tcolorbox}

\section{Zweckbestimmung}\label{sec:zweckbestimmung}
\begin{tcolorbox}
	Zweck des ganzen Systems beschreiben (Fließtext).
\end{tcolorbox}
Für die Abrechnungen der Gas-,Wasser- und Stromkosten müssen die Energieanbieter regelmäßig die Zählerstände ihrer
Kunden einfordern. Bisher war es Praxis diese per Post zu übermitteln oder diese online einzutragen. Wenn bei ersterem Fehler auftraten musste weiterer Briefverkehr erfolgen.\\
Der Zweck des Systems ist die Vereinfachung des beschriebenen Szenarios. Der Upload von Bildern durch die App ersetzt den Briefverkehr und spart den Arbeitsaufwand für das bearbeiten der Post ein. Administratoren können über das selbe Backend, an das die Kunden ihre Daten schicken, auf die Kundendaten zugreifen. Dies vermeidet Redundanzen und fehler beim Übertragen der Daten in die Datenbank.
Aus Kundensicht ist die Benutzung komfortabel und einfach, da die App schlicht und funktional ist. Das hochladen der Bilder ist einfach und benötigt nur geringen Aufwand. Außerdem ersetzen die Push-Notificationes die ansonsten per Brief geschickten Aufforderungen.


\section{Entwicklungsumgebung}\label{sec:entwicklungsumgebung}
\begin{tcolorbox}
	Oftmals treten neue Entwickler einem Projekt bei oder ein Entwicklungs-Rechner muss ersetzt werden.
	Daher sollen hier nennenswerte und grundlegende Frameworks, Bibliotheken, Tools und Sprachen notiert werden.
	Tabelle X stellt eine beispielhafte Umsetzung dar.
	Eine Unterteilung in Komponenten ist sinnvoll.
\end{tcolorbox}

\begin{table}[h]
	\centering
	\begin{tabularx}{\textwidth}{l l X}
		\rowcolor[HTML]{C0C0C0} 
		\textbf{Software} & \textbf{Version} & \textbf{URL} \\
		Java Development Kit & 8u144 & \url{http://www.oracle.com/technetwork/java/javase/downloads/index.html} \\
		\rowcolor[HTML]{E7E7E7} 
		Node & 12.9.0 & nodejs.org/en/download/current \\
		npm & 6.10.2 & npmjs.com \\
		\rowcolor[HTML]{E7E7E7} 
		React &16.9 &reactjs.org\\
		Software X & Version X & URL X \\
		\rowcolor[HTML]{E7E7E7} 
		Create React App & 3.1.1 & create-react-app.dev \\
		Software X & Version X & URL X \\
	\end{tabularx}
	\caption{Enwicklungsumgebung}
	\label{table:entwicklungsumgebung}
\end{table}