\section{Dokumentaufbau}\label{sec:dokumentaufbau}
\begin{tcolorbox}
	Inhalt und Struktur des vorliegenden Dokuments skizzieren (Fließtext).
\end{tcolorbox}

da die Karte verloren gehen könnte, der Brief nicht zugestellt werdenagen von Zählerstä 


\section{Zweckbestimmung}\label{sec:zweckbestimmung}
\begin{tcolorbox}
	Zweck des ganzen Systems beschreiben (Fließtext).
\end{tcolorbox}

Ziel des Dokuments: \\
Das Ziel dieses Dokuments ist die Dokumentation der Entwicklung des Onlineportals von adesso energy. 
Die enthaltenen Diagramme sollen neuen Teammitgliedern einen erleichterten Einstieg in die Struktur des Projekts ermöglichen 
und somit den Einarbeitungsprozess vor der aktiven Teilnahme am Entwicklungsprozess verkürzen und vereinfachen.  
Außerdem können sich Mitarbeiter, welche bereits am Projekt arbeiten, mit Hilfe dieser Entwurfsdokumentation einen Überblick über 
ihren aktuellen Stand und bei Unklarheiten ebenfalls über die Struktur des Projektes verschaffen.
Die frühe Planung des Zusammenspiels der einzelnen Anwendungsbereiche soll zu einem reibungslosen Ablauf 
in der tatsächlichen Entwicklung des Onlineportals führen und ermöglicht außerdem rechtzeitige Absprachen über die Funktionalität 
der Schnittstellen zwischen den verschiedenen Anwendungsbereichen.

\newpage

Aufbau des Dokuments: \\
In Kapitel 2 präsentieren wir die Aufteilung unseres Teams in Gruppen, welche sich auf die Entwicklung 
der verschiedenen Anwendungsbereiche spezialisiert haben. Daraufhin stellen wir in Kapitel 3 anwendungsbereichsspezifische 
Komponentendiagramme vor, welche die grobe Struktur der Anwendungsbereiche klären.
In Kapitel 4 folgt ein anwendungsbereichsübergreifendes Verteilungsdiagramm, welches die Verteilung der Komponenten auf die Hardware erläutert. 
In dem fünften Kapitel befindet sich eine Sammlung von Klassendiagrammen, die wiederum die Struktur des Entwurfes 
der jeweiligen Anwendungsbereiche spezifizieren. Anschließend werden in Kapitel 6 die wichtigsten Abläufe von Aufrufen 
der Methoden aus den Klassendiagrammen in Sequenzdiagrammen dargestellt, um ein besseres Verständnis 
des Zusammenspiels der Klassen zu schaffen. Schließlich ist das letzte Kapitel das Glossar. 
In diesem können die Definitionen von Grundbegriffen des Projektes nachgeschlagen werden, welche in 
dieser Entwurfsdokumentation ohne Erläuterung verwendet werden.
\\

\section{Entwicklungsumgebung}\label{sec:entwicklungsumgebung}
\begin{tcolorbox}
	Oftmals treten neue Entwickler einem Projekt bei oder ein Entwicklungs-Rechner muss ersetzt werden.
	Daher sollen hier nennenswerte und grundlegende Frameworks, Bibliotheken, Tools und Sprachen notiert werden.
	Tabelle X stellt eine beispielhafte Umsetzung dar.
	Eine Unterteilung in Komponenten ist sinnvoll.
\end{tcolorbox}

\begin{table}[h]
	\centering
	\begin{tabularx}{\textwidth}{l l X}
		\rowcolor[HTML]{C0C0C0} 
		\textbf{Software} & \textbf{Version} & \textbf{URL} \\
		Java Development Kit & 8u144 & \url{http://www.oracle.com/technetwork/java/javase/downloads/index.html} \\
		\rowcolor[HTML]{E7E7E7} 
		Node & 12.9.0 & \url{nodejs.org/en/download/current} \\
		npm & 6.10.2 & \url{npmjs.com} \\
		\rowcolor[HTML]{E7E7E7} 
		React &16.9 &\url{reactjs.org}\\
		Create React App & 3.1.1 & \url{create-react-app.dev} \\
		\rowcolor[HTML]{E7E7E7} 
		Spring & 5.1.6 & \url{projects.spring.io/spring-framework} \\
		Docker & 19.03.1& \url{www.docker.com} \\
		\rowcolor[HTML]{E7E7E7} 
		Android SDK API&LVL 23 & \url {https://developer.android.com/studio/releases/platforms\#\ 6.0} \\
		PostgreSQL &11.4 & \url{www.postgresql.org} \\
		\rowcolor[HTML]{E7E7E7}
		Storybook & 5.1 &\url {https://storybook.js.org} \\
		
		
	\end{tabularx}
	\caption{Enwicklungsumgebung}
	\label{table:entwicklungsumgebung}
\end{table}