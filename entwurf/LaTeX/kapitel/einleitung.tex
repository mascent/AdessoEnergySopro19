\section{Dokumentaufbau}\label{sec:dokumentaufbau}
Das Ziel dieses Dokuments ist die Dokumentation der Entwicklung des Onlineportals und der App von adesso energy. 
Die enthaltenen Diagramme sollen neuen Teammitgliedern einen erleichterten Einstieg in die Struktur des Projekts ermöglichen 
und somit den Einarbeitungsprozess vor der aktiven Teilnahme an der Entwicklung verkürzen und vereinfachen.  
Außerdem können sich Mitarbeiter, die bereits am Projekt arbeiten, mit Hilfe dieser Entwurfsdokumentation einen Überblick über 
ihren aktuellen Stand und bei Unklarheiten ebenfalls über die Struktur des Projektes verschaffen.
Die frühe Planung des Zusammenspiels der einzelnen Anwendungsbereiche soll zu einem reibungslosen Ablauf 
in der tatsächlichen Entwicklung des Onlineportals führen und ermöglicht außerdem rechtzeitige Absprachen über die Funktionalität 
der Schnittstellen.
\\\\
Das Dokument besteht aus 8 Kapiteln, die jeweilen Kapitel werden im Folgenden kurz beschrieben. In Kapitel \ref{chp:team-aufteilung} präsentieren wir die Aufteilung unseres Teams in Gruppen, welche sich auf die Entwicklung der verschiedenen Anwendungsbereiche spezialisiert haben. In Kapitel \ref{chp:rest-api} werden die REST-API Endpunkte beschrieben, sowie die verschiedenen DTO Datentypen erläutert. Daraufhin stellen wir in Kapitel \ref{chp:komponentendiagramme} anwendungsbereichsspezifische Komponentendiagramme vor, welche die grobe Struktur der Anwendungsbereiche klären. In Kapitel \ref{chp:verteilungsdiagramm} folgt ein bereichsübergreifendes Verteilungsdiagramm, welches die Verteilung der Komponenten auf die Hardware erläutert. 
Im \ref{chp:klassendiagramme}. Kapitel befindet sich eine Sammlung von Klassendiagrammen, die wiederum die Struktur des Entwurfes der jeweiligen Anwendungsbereiche spezifizieren. Anschließend werden in Kapitel \ref{chp:sequenzdiagramme} die wichtigsten Abläufe von Methodenaufrufe aus den Klassendiagrammen in Sequenzdiagrammen dargestellt, um ein besseres Verständnis des Zusammenspiels der Klassen zu schaffen. Schließlich ist das letzte Kapitel (\ref{chp:glossar}) das Glossar. 
In diesem können die Definitionen von Grundbegriffen des Projektes nachgeschlagen werden, welche in 
dieser Entwurfsdokumentation ohne Erläuterung verwendet werden. \newpage

\section{Zweckbestimmung}\label{sec:zweckbestimmung}
In diesem Abschnitt erklären wir kurz den Mehrwert, welchen unsere Software bietet. 
Für die Abrechnungen der Gas-,Wasser- und Stromkosten müssen Energieanbieter regelmäßig die Zählerstände ihrer
Kunden einfordern. Bisher war es Praxis diese per Post zu übermitteln oder online einzutragen. Wenn bei ersterem Fehler auftraten, musste weiterer Briefverkehr erfolgen.\\
Der Zweck des Systems ist die Vereinfachung des beschriebenen Szenarios. Der Upload von Bildern durch die App ersetzt den Briefverkehr und spart den Arbeitsaufwand für das Bearbeiten der Post ein. Administratoren können über das Back-End, an das die Kunden ihre Daten schicken, auf die Kundendaten zugreifen. Dies vermeidet Redundanzen und Fehler beim Übertragen der Daten in die Datenbank.\\
Aus Kundensicht ist die Benutzung komfortabel und einfach, da die App schlicht und funktional ist. Das Hochladen der Bilder ist einfach und erfordert nur geringe technische Kenntnisse. Außerdem ersetzen die Push-Notifications die, ansonsten per Brief geschickten, Aufforderungen den Zählerstand erneut abzulesen.


\section{Entwicklungsumgebung}\label{sec:entwicklungsumgebung}

\begin{table}[h]
	\centering
	\begin{tabularx}{\textwidth}{l l X}
		\rowcolor[HTML]{C0C0C0} 
		\textbf{Software} & \textbf{Version} & \textbf{URL} \\
		Java Development Kit & 8u144 & \url{http://www.oracle.com/technetwork/java/javase/downloads/index.html} \\
		\rowcolor[HTML]{E7E7E7} 
		Node & 12.9.0 & \url{nodejs.org/en/download/current} \\
		npm & 6.10.2 & \url{npmjs.com} \\
		\rowcolor[HTML]{E7E7E7} 
		React &16.9 &\url{reactjs.org}\\
		Create React App & 3.1.1 & \url{create-react-app.dev} \\
		\rowcolor[HTML]{E7E7E7} 
		Spring & 5.1.6 & \url{projects.spring.io/spring-framework} \\
		Docker & 19.03.1& \url{www.docker.com} \\
		\rowcolor[HTML]{E7E7E7} 
		Android SDK API&LVL 23 & \url {https://developer.android.com/studio/releases/platforms\#\ 6.0} \\
		PostgreSQL &11.4 & \url{www.postgresql.org} \\
		\rowcolor[HTML]{E7E7E7}
		Android Studio & 3.5 & \url{https://developer.android.com/studio} \\
		Storybook & 5.1 &\url {https://storybook.js.org} \\
	\end{tabularx}
	\caption{Enwicklungsumgebung}
	\label{table:entwicklungsumgebung}
\end{table}