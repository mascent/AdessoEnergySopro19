
In diesem Kapitel sind die Sequenzdiagramme beschrieben, die Vorgänge beschreiben bei denen das Verhalten nicht trivial ist.
Nicht trivial ist ein Verhalten bei dem das Diagramm nicht einem ``durchreichen'' von Aufrufen entspricht.
\section{Triviale Abfrage}
\begin{figure}[H]
	\centering
	\includegraphics[width = 12cm]{img/diagrams/TrivialDiagram}
	\caption{Abfrage der Nutzerlisten}
\end{figure}
Als Beispiel für ein triviales Sequenzdiagramm lässt sich das Abfragen der Daten durch einen Admin hernehmen. Dabei beschreibt ``User [0..*]'' die Rückgabe eines ``UserList DTO''.
 
\section{Bilder in der App übermitteln}
\begin{figure}[H]
	\centering
	\caption{Erfolgreiche Bildübermittlung}
	\includegraphics[width=16cm]{img/diagrams/SubmitFotoSequence}
\end{figure}
Das folgende Sequenzdiagramm beschreibt den Vorgang des erfolgreichen Übermitteln eines \\Zählerstandes als Bild.
Zuerst läd der Nutzer ein Foto hoch, dieses wird mittels der Methode azureAnalyse vom Netzwerk-Controller per Post-Request ans Backend der Website geschickt. Dieses leitet das Bild per HTTP an die Klassifizierungsschnittstelle des AzureWrappers weiter. Nach der Antwort wird über HTTP das Bild an die nächste Schnittstelle Area-Detect geschickt. Als letztes wird das Bild noch per HTTP an die Schnittstelle Parse-Text geschickt. Die erhaltenenen Daten werden jetzt einer ersten Plausibilitätsprüfung unterzogen. (Richtigkeit der Stellen und Klassifizierung) \\
Nun wird mittels getUsersMeters() alle Zähler für den Nutzer von der Datenbank abgefragt, um zu Prüfen, ob die Nummer des Zählers in enthalten ist.
Wenn dies der Fall ist wird eine HTTP Response an den Network-Controller gesendet, die die gelesene Zählernummer und den gelesenen Zählerstand übermittelt. Nun wird der Nutzer aufgefordert den gelesenen Wert zu bestätigen. \\
Die Bestätigung entspricht dabei einer manuellen Eingabe.

\section{Zählerstände für User updaten}
\begin{figure}[H]
	\centering
	\includegraphics[width = 15cm]{img/diagrams/ChangeReading}
	\caption{Abfrage der Nutzerlisten}
\end{figure}
Administratoren können unter Angabe eines Grundes Zählerstände aktualisieren.
Dafür muss der Administrator zuerst mittels eines HTTP-Requests zu einer User-ID alle Zähler vom User Controller abfragen. Danach werden mittels einer weiteren Anfrage alle Zählerstände des Zählers abgefragt. Nun kann mittels eines PUT-Request mit der rid ein bestimmter Eintrag verändert werden. Dafür wird der neue Zählerstand, der Grund der Änderung und die ID des Admins versendet, die dann für den Funktionsaufruf ''updateReading'' verwendet wird. Wenn die Änderung erfolgreich war, sendet der Reading Controller in der HTTP Response einen Erfolg zurück.